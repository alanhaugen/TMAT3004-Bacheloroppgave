\section*{Sammendrag}

%Lengre oppgaver, som bachelor- og masteroppgaver, skal ha et sammendrag. Det er viktig at sammendraget er informativt, siden det skal kunne leses av lesere som ikke er eksperter på området.

%Sammendraget skal være kort, helst ikke over én A4-side, og skal gi et overblikk over hovedinnholdet. Du skal fortelle leseren:

%hva du har undersøkt

%Dette prosjektet har gått ut på å bruke maskinlæring til å lage en maskin som kan skape en oversikt over mengden fisk av typen torsk og sei som beiter rundt oppdrettsanlegg. Maskinen lager en tidsoversikt over fordelingen av fiskeartene. Dette skal hjelpe forskere med å studere fenomenet fôrsprengt fisk. Prosjektet er gjort i samarbeid med Nofima.%Fiskere vil ha svar på om fôrsprengt fisk, også kjent som pellets-fisken, kan være farlig.

%Fôrsprengt fisk har blitt vanlig i Lyngen kommune i Troms. ``Pellets-fisk” ble vanlig etter at de begynte med oppdrett i fjorden. Fisken er full av fôr og stinker. Den er unaturlig og tjukk.

%``Pellets-fisk'' er villfisk som har spist oppdrettsfôr som inneholder medisin. Denne fisken blir omsatt. Et eksempel er når lakselusmiddelet emamectin ble gitt til oppdrettsfisk i tre uker i 2017 i denne fjorden i Lyngen. Oppdrettsfisken var i karantene i denne perioden, og blir ansett som giftig. I samme periode leverte fiskere villfisk fra denne fjorden til et mottak i Troms. Den var full av pellets. Det sto ut av munnen på både sei og torsk.


%hva du har undersøkt

% Bruk en an inngang til avsnittet, du skal gi et sammendrag av hva BO oppgaven var, hvordan den ble løst og hva resultatet ble.

Denne bacheloroppgaven handler om å bruke maskinsyn til å telle villfisk som beiter under oppdrettsmerder. Forsøket ble gjort som en del av Sameksistensprosjektet til Nofima. Objektdetekteringen baserer seg på dype nevrale nettverk og er i stand til å klassifisere og telle fiskene torsk og sei.

Kunstige nevrale nettverk kalles for modeller. Maskinlæring har blitt et populært paradigme innenfor informatikk. Modeller lærer fra data. YOLOv4 og RetinaNet er objektdetekteringsmodeller.

%hvordan du gjorde det
YOLOv4 og RetinaNet ble anvendt til fiskedeteksjon. Begge modellene brukte CNN-konsepter hentet fra forhåndstrente nettverk. Det ble utviklet et datasett basert på video fra og ved lagringsmerder av fisk. Gjennomsnittlig presisjon (AP50) for RetinaNet ble 66,3 \%. En gjennomsnittlig objektdeteksjonspresisjon (mAP) lik 73 \% ble oppnådd for YOLOv4-modellen.

%hva du fant ut
Dette forsøket viser at maskinsyn kan anvendes til å løse oppgaver innenfor akvakultur. Maskinsyn kan anvendes til å finne omfanget av torsk og sei som beiter rundt oppdrettsanlegg.%, og kan muligens anvendes til å detektere rømninger fra merder samt utvikles videre til å detektere lakselus på fisk i merdene.

Skriptene til å reprodusere forsøket er offentlig tilgjengelig fra: \\ \url{https://github.com/alanhaugen/TMAT3004-Bacheloroppgave}.

\section*{Abstract}

This bachelor thesis describes a method for detecting and counting the wild fish feeding in the vicinity of fish farms by using computer vision. The work was done as a part of Nofima's Coexistence project. The fish detection is based on deep neural networks and can detect and count Atlantic cod and saithe.

Artificial neural networks are called models. Machine learning has become a popular new programming paradigm. Models learn from data. YOLOv4 and RetinaNet are examples of object detection models.

The networks YOLOv4 and RetinaNet were used for fish detection. Both of the models used CNN concepts from pre-trained networks. A dataset based on videos of fish from and around fish farms was generated. The average precision (AP50) for the RetinaNet model was 66,3 \%. A mean average precision (mAP) of 73 \% was achieved for the YOLOv4 model.

The study demonstrates that computer vision is a viable technology for solving problems within the aquaculture industry. Computer vision can be used to find the amount of Atlantic cod and saithe grazing in the vicinity of fish farms.

Scripts to reproduce this study are made publicly available at: \\ \url{https://github.com/alanhaugen/TMAT3004-Bacheloroppgave}.

%Det er et ønske å kartlegge hvor mange villfisk som trekker til oppdrettsanlegg, og under hvilke forhold. Sjømatdivisjonen og Akvadivisjonen ved Nofima har en strategisk internsatsing å samle kunnskap om sameksistens mellom ulike marine næringer og interesser, deriblant hvordan man kan unngå konflikter mellom ulike interesser.

%hvordan du gjorde det

%Maskinlæring er når datamaskiner lærer fra data. I 2012 vant AlexNet Stanford University sin årlige ImageNet bildeklassifiseringskonkurranse. Teamet til Alex anvendte deep learning og det har revolusjonert maskinsyn og kunstig intelligens. Flere modeller har blitt laget siden AlexNet, og bildeklassefisering, objektdeteksjon og segmentering har blitt mye mer nøyaktig nå som de anvender dype kunstige neurale nettverk.

%Facebook introduserte RetinaNet i 2017, og i 2018 kom YOLOv3. Det er disse teknologiene som har blitt anvendt til å lage en modell som kan kjenne igjen torsk og sei med ...(resultater) .

%hva du fant ut

%Prosjektet viser at maskinsyn kan anvendes til å løse oppgaver innenfor akvakultur. Teknologien som presenteres i dette dokumentet kan anvendes til å finne omfanget av torsk og sei som beiter rundt oppdrettsanlegg, og kan muligens anvendes til å detektere rømninger fra merder samt utvikles videre til å detektere lakselus på fisk i merdene.

%Ved å lese sammendraget skal leserne kunne avgjøre om de er interessert i å lese resten av teksten.