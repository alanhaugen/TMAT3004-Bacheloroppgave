\section{Innledning}

%I innledningen skal du plassere deg i fagfeltet og vise at du har kjennskap til tidligere forskning. Innledningen skal gjøre rede for hva vi vet og hva vi ikke vet om feltet.

%Dette gjør du ved å presentere:

%et problem eller et fenomen du skal studere

Fôrsprengt fisk beskrives i detalj i del \ref{part:background}. «Pellets-fisk» er villfisk som spiser fôrpellets fra under oppdrettsmerdene. Se figur \ref{fig:anlegg}. Fiskere ønsker svar på om villfisk som spiser oppdrettsfôr som inneholder medisin kan være farlig. Oppdrettsfisk er i karantene når de medisineres, men villfisk spiser fortsatt fôret og blir omsatt.

For å lage et dataprogram som kan automatisk telle og artsbestemme fisk så ble maskinlæring, en form for kunstig intelligens, anvendt. Teorien beskrives i del \ref{part:ai}. Tidligere vitenskapelige forsøk der forskere har brukt maskinlæring til fiskeklassifisering og fiskedeteksjon nevnes i del \ref{part:prev_work}. 

I del \ref{part:method} presenteres metoden som ble anvendt til å gjøre automatisk analyse av video. Det er først og fremst norsk torsk, $Gadus$ $morhua$, og sei, $Pollachius$ $virens$, som man ønsker en oversikt over. Et datasett av torsk og sei skulle utvikles for å trene opp maskinlæringsmodeller, måten dette ble gjort er beskrevet i del \ref{part:data}. Det ble planlagt å filme torsk og sei i gråtoner, men koronapandemien av 2020 gjorde det vanskelig å få tilgang til data. I del \ref{part:training} blir fremgangsmåten som ble brukt til å trene opp deep learning-modeller som kan kjenne igjen torsk og sei beskrevet. I del \ref{part:opencv} forklares en av måtene en kan lage et dataprogram med en deep learning-modell som kan gi en tidsoversikt over mengden torsk og sei. Nøyaktigheten til modellene ble målt, resultatene presenteres i del \ref{part:results}.

Det er mulig å forbedre dataprogrammet som har blitt utviklet til å gjenkjenne torsk og sei. Nøyaktigheten til et objektdeteksjonssystem blir gitt som mAP (mean Average Precision). Modellen til dataprogrammet har en mAP på 73 \%. En forbedret modell vil kunne gi oppdrettsnæringen en enda bedre tidsoversikt over torsk og sei under oppdrettsanleggene. En drøfting av resultatene er gitt i del \ref{part:discussion}.

Siste del er en konklusjon, del \ref{part:end}. Maskinsyn kan anvendes til å hjelpe oppdrettsnæringen med å få kontroll på fôring av oppdrettfisk ved å gi næringen en tidsoversikt over mengden torsk og sei som trekker til anleggene.

\begin{figure} 
\begin{center} 
\includegraphics[scale=0.7]{figures/merder-fisk}
\caption{\small \sl Oppdrettsbransjen er i konflikt med fiskere som mener oppdrettsnæringen ødelegger fiskeplasser i fjordene med anlegg og reduserer kvaliteten til villfisk som spiser fôrpellets under merdene \cite{Olsen m.fl. 2018}. Figur: Spruill \cite{Spruill 2011 s. 12}. \label{fig:anlegg}} 
\end{center} 
\end{figure} 

Bacheloroppgaven er gjort i samarbeid med Nofima som en del av Sameksistensprosjektet \cite{Robertsen 2020}.