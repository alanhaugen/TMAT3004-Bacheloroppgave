%Om avslutningen din skal være en konklusjon eller en oppsummering, avhenger av problemstillingen din. En konklusjon skal svare på problemstillingen, mens en oppsummering gjentar det viktigste fra oppgaven. Det er ikke uvanlig å velge en kombinasjon av de to, hvor du både oppsummerer oppgaven kort, men også svarer på problemstillingen.

%Det er lurt å la avslutningen speile innledningen, ved å si hva du har gjort. Avslutningen bør også sette oppgaven din i et større perspektiv, og peke på hvilke muligheter du ser ut fra ditt prosjekt. Hvilke bidrag har din undersøke gitt til faget? Er det noe som burde blitt studert ytterligere? Slik tar du utgangspunkt i ditt eget prosjekt og peker på mulighet for oppfølging.

\section{Konklusjon}

\label{part:end}

Maskinsyn kan anvendes til å hjelpe oppdrettsnæringen med å få kontroll på fôring av oppdrettfisk ved å gi industrien en tidsoversikt over mengden torsk og sei som trekker til anleggene.

Ved å tilpasse fôringen av oppdrettsfisken så vil fôrsprengt fisk bli et mindre problem. Kunnskapen som kan samles med automatisk videoanalyse kan hjelpe de marine næringene med å leve godt sammen langs kysten.

Maskinlæringsmodellene kan utvikles videre. En forbedret modell vil kunne gi oppdrettsnæringen en enda bedre tidsoversikt over torsk og sei under oppdrettsanleggene.

Et forbedret datasett, med flere fiskearter og mindre bias vil også forbedre resultatene. Automatisk analyse av undervannsvideoer kan gi forskere regelmessig informasjon om de marine økosystemene. Det er viktig å ha denne informasjonen når en gjør avgjørelser som kan påvirke miljøet og økonomien til de marine næringene.
