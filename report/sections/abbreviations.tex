\section*{Forkortelser og faguttrykk}

\begin{description}

\item[Algoritme] er en sekvens av instruksjoner som datamaskiner følger for å prosessere data
\item[AP] Average Precision
\item[API] Application Programming Interface
\item[Azure] nettskytjeneste fra Microsoft der en kan leie en Nvidia Titan-GPU
\item[Backpropagation] er en algoritme for å trene maskinlæringsmodeller
\item{Bias} urettferdig vekting i modellen
\item[COCO] er et datasett fra Microsoft med 80 klasser
\item[CPU] Central Processing Unit
\item[Darknet: Open-Source Neural Networks in C] er et maskinlæringsrammeverk fra Joseph Redmon
\item[Deep Learning] er en form for maskinlæring der mange kerneler er koblet sammen
\item[Detectron2] er et objektdetekteringssystem fra Facebook som bruker PyTorch
\item[dnn] Deep Neural Networks
\item[FPS] Frames Per Second
\item[GPU] Graphical Processing Unit, den kan trene maskinlæringsmodeller
\item[IDE] Integrated Development Environment
\item[ImageNet] er et generelt datasett fra Stanford University
\item[Inference] eller inferens kalles også for en 'forward pass', det er når en modell forutser hva som finnes i et bilde eller en prøve
\item[IoU] Intersection over Union, mål på objektdetekteringsnøyaktighet
\item[Klasse] er også kjent som target eller label, det er en merkelapp som beskriver en bounding-box eller objekt
\item[Kernel] også kjent som filter, kunstig nevron eller perceptron er en funksjon som består av vekter og bias, den tar inn matriser, abstrakte konsepter eller «features», og gir fra seg nye abstrakte konsepter, nye «features», basert på innmatningen
\item[Kildekode] er algoritmene, instruksjonene, eller «DNA-et», til et dataprogram
\item[Maskinlæringsbiblioteker] brukes ofte som synonymt med maskinlæringsrammeverk
\item[KI] Kunstig Intelligens
\item[Kunstige nevrale nettverk] se modell
\item[Kunstig nevron] se kernel
\item[mAP] mean Average Precision, en måling på objektedekteringsnøyaktighet
\item[Maskinsyn] har i nyere tid med utviklingen innenfor kunstige nevrale nettverk (aka “deep learning”) ført til maskiner som kan med høy nøyaktighet detektere og klassifisere objekter%Computer Vision has become ubiquitous in our society, with applications in search, image understanding, apps, mapping, medicine, drones, and self-driving cars. Core to many of these applications are visual recognition tasks such as image classification, localization and detection. Recent developments in neural network (aka “deep learning”) approaches have greatly advanced the performance of these state-of-the-art visual recognition systems. This course is a deep dive into details of the deep learning architectures with a focus on learning end-to-end models for these tasks, particularly image classification. During the 10-week course, students will learn to implement, train and debug their own neural networks and gain a detailed understanding of cutting-edge research in computer vision. The final assignment will involve training a multi-million parameter convolutional neural network and applying it on the largest image classification dataset (ImageNet). We will focus on teaching how to set up the problem of image recognition, the learning algorithms (e.g. backpropagation), practical engineering tricks for training and fine-tuning the networks and guide the students through hands-on assignments and a final course project. Much of the background and materials of this course will be drawn from the ImageNet Challenge.
\item[ML] Machine Learning, når en datamaskin lærer fra data
\item[Modell] er blant annet et annet ord for kunstige nevrale nettverk, en modell er trent på data, og gitt lignende data så kan en modell inferere hva den nye dataen er 
\item[Nettsky] er en tjeneste der en kan leie en kraftig datamaskin, slik som en GPU-server
\item[Nøyaktighet] defineres som True positives / (True positives + False positives)
\item[Overfitting] overtilpasning er når nettverket kan detektere objekter på bilder fra treningsdatasettet, men kan ikke detektere objekter på noen andre bilder
\item[Optical flow] er bevegelsene mellom bildene i en film
\item[R-CNN] Region-Based Convolutional Neural Network
\item[Perceptron] se kernel
\item[Presisjon] defineres som True positives / (True positives + False negatives)
\item[Programvarebibliotek] er kildekode som kan anvendes til å lage dataprogrammer, det skiller seg fra programvarerammeverk ved at programmerere har mer kontroll over arkitekturen til dataprogrammet som anvender programvarebibliotek
\item[Programvarerammeverk] er programvare som kan konfigureres til å lage modeller eller nye programmer
\item[Programvareutgivelse] er typisk et Unix-miljø for programmerere som gjør det lettere å programmere datamaskiner
\item[PyTorch] er et maskinlæringsrammeverk fra Facebook
\item[RetinaNet] er en objektdetekteringsmodell fra Facebook som kan brukes sammen med Detectron2
\item[Real-time] sanntid
\item[Skript] er et annet ord for kildekode, det brukes vanligvis om koden til enklere programmeringsspråk slik som Python
\item[ssh] Secure SHell
\item[STL] Standard Template Library
\item[SVM] Support Vector Machine, en metode som baserer seg på kvadratisk optimering
\item[TL] Transfer learning
\item[Unix] er et avansert operativsystem for programmerere
\item[YOLO] You Only Look Once, real-time objektdeteksjon


\end{description}