\subsection{Innledning -- bakgrunn for oppgaven}
\subsection{Teori}
\subsubsection{Introduksjon til kunstig intelligens}

\subsubsection{Maskinlæring}

Machine Learning: Learning from data

Data is king.

Data collection, annotation, preporation etc.

Data > Algorithm > Training > Evaluation > Deployment > Predictions

	Gather data from every legal source possible (public data sets, purchase data, collect data, synthesize data (super poweful))

	Manually check data
	Look for biases
	Look for insights
	Clean up

	Iterative: Partition data 60 (training)/20 (testing accuracy training)/20 (test)

Model / Algorithm

	Image classification
	Object detection 
	Segmentation

	Constraints

	Experimentation (test multiple viable models)

Training

	Data augmentation
	Training parameter (optimizer, rate etc.)
	Visualizsation (check if it is going correctly)

Evaluation
	
	Test. Check model size, speed and ACCURACY

Deployment

	Optimizations, deploy, feedback (know when it went badly, check failed images)	

\subsubsection{Neural networks}

Classification (Supervised learning)

	Seperating data into groups

	Binary classification (two groups) (Sigmoid activation is used)

	Multiclass classification (Activation: Softmax) (Loss function: Cross entropy loss)

	Regression (Activation: Linear) (Loss function: MSE loss)

	Decision boundary seperates the groups by the decision function

	Training is learning the decision function

	Deciding decision function is called training

	Data is on a plane (2D) or a hyperplane (higher dimensions)

	Input layer > Hidden layer (can be many layers) > output

	Each layer (node) in a nn is a neuron or perceptron

	Perceptron: Calculate weighted sum of inputs and add bias. Then apply activation function (non-linear)

	Every layer looks for a pattern found in the previous layer. If it is found, it "fires up"

	An example of an activation function is ReLU (Rectified Linear Unit)

	Another example is the sigmoid function, and tanh

	An activation function creates non-linearity

	The number of hidden layers is called the networks depth (depth = 2 is typical for simple problems)

Loss functions

	Classification outputs a category (class)

	Regression outputs numerical values (or a vector of numerical values)

	Many problems are optimizatino problems in ML, either to minimize or maximaze a value of a function

	These functions are called the objective function

	When finding the minimum, it is called a loss, or cost, function

	e = y - ^y (error is ground thruth minus model output)

	An error, L, can be considered either a square (MSE, most common) or an absolute number (MAE, when data has many outliers)

Single layer perceptron kan løse lineære problemer

Ved å gjøre "feature engineering" så kan ikke-lineære problemer løses

Deep learning gjør at en kan løse lineære problemer om en bruker ikke-lineær aktivering. ReLU konvergerer raskt.

\subsubsection{Maskinsyn med OpenCV}
\subsubsection{Video med undervannskamera fra merdene}
\subsubsection{Analysere video}
\subsubsection{Deep Learning med OpenCV}
\subsubsection{PyTorch}
\subsubsection{Segmentere ut fisk}
\subsubsection{Object Detection med OpenCV}
\subsubsection{Object Tracking med OpenCV}
\subsubsection{Klassifisere hver fisk etter art}
\subsubsection{Registrere antall individer av hver art fortløpende}
\subsection{Praktisk gjennomføring}
\subsubsection{Programvareutvikling med maskinlæring implementert i C++}
\subsubsection{Videostrøm fra merdene}
\subsection{Resultater}
\subsection{Diskusjon}
\subsection{Konklusjon}
\subsection{Referanseliste}
