% Vurdering 1/5
%Du skal her drøfte resultatene dine og sette dem inn i en sammenheng. Å drøfte vil si å:

%sette ulike synspunkter, momenter, argumenter, faktorer og årsaker opp mot hverandre
%vurdere og sette dem opp mot hverandre. Finnes det flere ulike tolkninger av resultatene?
%Du må i kapitlet svare på:

%hvordan svarer resultatene dine på forskningsspørsmålene dine?
%hva betyr resultatene?
%Et lurt tips er å gjenta forskningsspørsmålene, slik at leseren blir minnet på hva de er.

%Du skal òg se tilbake på studien og vurdere hvor gyldig og pålitelig den har vært.

%Hva kunne blitt gjort annerledes?
%Hva er sterke og svake sider?
%Her kan du for eksempel kritisere metodene som du har brukt, og forklare hva du kunne gjort annerledes.

\section{Diskusjon}

Fish4Knowledge, LifeCLEF og SeaCLEF datasettene kan legges til modellen, da vil den kjenne igjen flere fiskearter.

Overfôring

The most important outcome of this research is the classification accuracy achieved. With the other species class included, the accu- racy of the classification is 89.0 \%, which is competitive with or exceeds other recently reported results on fish species identifica- tion tasks (Hsiao et al., 2014; Huang et al., 2015; Salman et al., 2016). If the other species class is excluded, the classification accu- racy within the 16 species classes is 94.3 \%, which is competitive with the identification rates of human experts for similar tasks (Culverhouse et al., 2003).